% Search for all the places that say "PUT SOMETHING HERE".

\documentclass[11pt]{article}
\usepackage{amsmath,textcomp,amssymb,graphicx,enumerate,hyperref,enumitem,mathtools,tikz-qtree,listings,chemformula,bm,graphicx,grffile,gensymb,physics,amssymb,datetime,siunitx}
\graphicspath{{/Users/jonathansun5/Documents/Fall 2017/MCB 166/Homeworks/HW 3/Screen Shot 2017-10-14 at 6.38.44 PM.png} {/Users/jonathansun5/Documents/Fall 2017/MCB 166/Homeworks/HW 3/Screen Shot 2017-10-14 at 6.40.12 PM.png} {/Users/jonathansun5/Documents/Fall 2017/MCB 166/Homeworks/HW 3/Screen Shot 2017-10-14 at 7.07.44 PM.png} {/Users/jonathansun5/Documents/Fall 2017/MCB 166/Homeworks/HW 3/Screen Shot 2017-10-14 at 7.10.16 PM.png}}

\makeatletter
\newcommand{\leqnos}{\tagsleft@true\let\veqno\@@leqno}
\newcommand{\reqnos}{\tagsleft@false\let\veqno\@@eqno}
\reqnos
\makeatother

\def\Name{Jonathan Sun}  % Your name
\def\SID{25020651}  % Your student ID number
\def\Homework{3} % Number of Homework
\def\Session{Fall 2017}


\title{MCB166 --- \Session --- \Homework}
\author{\Name, SID \SID}
\markboth{MCB166 --- \Session --- \Homework --- \Name}{MCB166 --- \Session --- \Homework --- \Name}
\pagestyle{myheadings}
\newdate{date}{17}{10}{2017}
\date{\displaydate{date}}

\def\endproofmark{$\Box$}
\newenvironment{proof}{\par{\bf Proof:}}{\endproofmark\smallskip}

\usepackage[margin=1in]{geometry}



\begin{document}
\maketitle

\newpage
\begin{enumerate}[label=\arabic*.]
\item
The unicellular organism \textit{Paramecium caudatum} shows a resting potential (RP) and an action potential (AP) that are similar in many respects to corresponding neural potentials. With the cell in ``typical pond water,'' the following measurements were made with an intracellular electrode:
\begin{center}
\includegraphics[width=0.75\textwidth]{/Users/jonathansun5/Documents/Fall 2017/MCB 166/Homeworks/HW 3/Screen Shot 2017-10-14 at 6.38.44 PM.png}
\end{center}
If one varies $[\ch{K+}]_{\text{out}}$ only, or $[\ch{Ca^{2+}}]_{\text{out}}$ only, one observes the following:
\begin{center}
\includegraphics[width=0.75\textwidth]{/Users/jonathansun5/Documents/Fall 2017/MCB 166/Homeworks/HW 3/Screen Shot 2017-10-14 at 6.40.12 PM.png}
\end{center}
In the following questions, assume that the membrane of \textit{P. caudatum} is normally permeable only to \ch{K+}, \ch{Ca^{2+}}, and water.
\begin{enumerate}[label=(\alph*)]
\item
In the resting state, which of these is true? Explain concisely.
\begin{enumerate}[label=\roman*.]
\item
$P_{\ch{K}} > P_{\ch{Ca}}$
\item
$P_{\ch{K}} = P_{\ch{Ca}}$
\item
$P_{\ch{K}} < P_{\ch{Ca}}$
\end{enumerate}







\item
Which is true during the peak of the AP? Explain concisely.









\item
Compared to the ionic concentrations of ``typical pond water,'' is $\ch{K+}_{\text{in}}$ greater than, equal to, or less that $\ch{K+}_{\text{out}}$? Explain.










\item
Compare also $\ch{Ca^{2+}}_{\text{in}}$ with $\ch{Ca^{2+}}_{\text{out}}$.











\item
When the posterior end of the organism is mechanically tapped, the membrane transiently hyperpolarizes. What permeability change(s) might be responsible? Explain.











\end{enumerate}





\newpage
\item
\begin{enumerate}[label=(\alph*)]
\item
Briefly state the assumptions for the constant field model.



\item
Sketch approximately the \textit{I-V} relations predicted by the constant field model for various ratios of intracellular and extracellular ion concentrations, i.e., when $\frac{[\ch{C}]_{\text{in}}} {[\ch{C}]_{\text{out}}} = 0\text{, } 0.1\text{, } 1\text{, } 30\text{, or } \infty$.



\item
Using the data provided in the figure below, calculate the ratio of $P_{\ch{Na}} / P_{\ch{K}}$ that predicts the resting potential as a function of $[\ch{K}]_{\text{out}}$ for the \textit{Myxicola} neuron. Note: $[\ch{Na+}]_{\text{out}} = 430 \text{mM}$, $[\ch{Na+}]_{\text{in}} = 12 \text{mM}$, $[\ch{K+}]_{\text{in}} = 270 \text{mM}$, and $P_{\ch{Cl}} = 0$.
\begin{center}
\includegraphics[width=0.75\textwidth]{/Users/jonathansun5/Documents/Fall 2017/MCB 166/Homeworks/HW 3/Screen Shot 2017-10-14 at 7.07.44 PM.png}
\end{center}



\end{enumerate}













\newpage
\item
The instantaneous and steady-state $I-V$ relations of a neuron obtained from voltage-clamp experiments are shown below:
\begin{center}
\includegraphics[width=0.75\textwidth]{/Users/jonathansun5/Documents/Fall 2017/MCB 166/Homeworks/HW 3/Screen Shot 2017-10-14 at 7.10.16 PM.png}
\end{center}
The time-dependent current follows first-order kinetics with the time constant $\tau = 0.1$ sec.
\begin{enumerate}[label=(\alph*)]
\item
Draw the membrane current with respect to time after the membrane voltage is stepped from $V_H = -60 \text{mV}$ to $V_c = 0 \text{mV}$ and to $V_c = -80 \text{mV}$. Label the current and time axes with the appropriate units.






\item
Repeat (a) after the membrane voltage is stepped from $V_H = +50 \text{mV}$ to $V_c = +100 \text{mV}$.












\end{enumerate}




\end{enumerate}
\end{document}
