% Search for all the places that say "PUT SOMETHING HERE".

\documentclass[11pt]{article}
\usepackage{amsmath,textcomp,amssymb,graphicx,enumerate,hyperref,enumitem,mathtools,tikz-qtree,listings,chemformula,bm,graphicx,grffile,gensymb,physics,amssymb,datetime,siunitx,multicol,pgfplots}
\graphicspath{{}}
\pgfplotsset{compat=1.14}
\makeatletter
\newcommand{\leqnos}{\tagsleft@true\let\veqno\@@leqno}
\newcommand{\reqnos}{\tagsleft@false\let\veqno\@@eqno}
\reqnos
\makeatother

\def\Name{Jonathan Sun}  % Your name
\def\SID{25020651}  % Your student ID number
\def\Homework{Final Exam} % Number of Homework
\def\Session{Fall 2017}


\title{MCB166 --- \Session --- \Homework}
\author{\Name, SID \SID}
\markboth{MCB166 --- \Session --- \Homework --- \Name}{MCB166 --- \Session --- \Homework --- \Name}
\pagestyle{myheadings}
\newdate{date}{11}{12}{2017}
\date{\displaydate{date}}

\def\endproofmark{$\Box$}
\newenvironment{proof}{\par{\bf Proof:}}{\endproofmark\smallskip}

\usepackage[margin=1in]{geometry}



\begin{document}
\maketitle

\newpage
\begin{enumerate}[label=\arabic*.]
\item
\textbf{Membrane potentials in a retinal rod}
\vspace*{1\baselineskip}
\\
Consider a membrane permeable to \ch{Na+}, \ch{K+}, and \ch{Cl-} with relative permeabilities, $P_{\ch{Na}}$, $P_{\ch{K}}$, and $P_{\ch{Cl}}$. Assume the ions all obey the constant-field current voltage curve
\begin{align*}
I_x = F P_x Z_x v \left([X]_o - [X]_i \text{exp}(v)\right) / (1 - \text{exp}(v)).
\end{align*}
Here $v = eV / kT = V / 25 \text{mV}$, $X$ is the concentration of species $x$ in mM, $F$ is the Faraday constant, and $Z_x$ is the valence of species $x$.
\vspace*{1\baselineskip}
\\
We want to compare current-voltage relations and reversal potentials for two different ion channels. One is the typical imperfectly-selective potassium channel (\ch{K}-ch), for which $\alpha_{\ch{K}} = P_{\ch{K}} / P_{\ch{Na}} = 50$. The other is a cation channel (Cat-ch), such as is found in postsynaptic and sensory-receptor membranes, for which $\alpha_U = P_{\ch{K}} / P_{\ch{Na}} = 1$.
\vspace*{1\baselineskip}
\\
For a vertebrate photoreceptor, the internal and external ion concentrations are:
\begin{align*}
[\ch{K}]_o = 5 \text{mM; } [\ch{Na}]_o = 120 \text{mM;} \\
[\ch{K}]_i = 125 \text{mM; } [\ch{Na}]_i = 12 \text{mM.}
\end{align*}
\begin{enumerate}[label=(\alph*)]
\item
ok.
\vspace*{1\baselineskip}
\\
ok.



\item
ok.
\vspace*{1\baselineskip}
\\
ok.



\item
ok.
\vspace*{1\baselineskip}
\\
ok.



\item
ok.
\vspace*{1\baselineskip}
\\
ok.



\item
ok.
\vspace*{1\baselineskip}
\\
ok.



\end{enumerate}



\newpage
\item
Ok.
\begin{enumerate}[label=(\alph*)]
\item
ok.
\vspace*{1\baselineskip}
\\
ok.



\item
ok.
\vspace*{1\baselineskip}
\\
ok.




\item
ok.
\vspace*{1\baselineskip}
\\
ok.



\end{enumerate}



\newpage
\item
Ok.
\begin{enumerate}[label=(\alph*)]
\item
ok.
\vspace*{1\baselineskip}
\\
ok.



\item
ok.
\vspace*{1\baselineskip}
\\
ok.



\item
ok.
\vspace*{1\baselineskip}
\\
ok.
\end{enumerate}



\newpage
\item
Ok.
\begin{enumerate}[label=(\alph*)]
\item
ok.
\vspace*{1\baselineskip}
\\
ok.




\item
ok.
\vspace*{1\baselineskip}
\\
ok.




\item
ok.
\vspace*{1\baselineskip}
\\
ok.






\item
ok.
\vspace*{1\baselineskip}
\\
ok.

\end{enumerate}



\newpage
\item
Ok.
\begin{enumerate}[label=(\alph*)]
\item
ok.
\vspace*{1\baselineskip}
\\
ok.



\item
ok.
\vspace*{1\baselineskip}
\\
ok.





\item
ok.
\vspace*{1\baselineskip}
\\
ok.





\item
ok.
\vspace*{1\baselineskip}
\\
ok.
\end{enumerate}




\newpage
\item
Ok.
\begin{enumerate}[label=(\alph*)]
\item
ok.
\vspace*{1\baselineskip}
\\
ok.



\item
ok.
\vspace*{1\baselineskip}
\\
ok.





\item
ok.
\vspace*{1\baselineskip}
\\
ok.





\item
ok.
\vspace*{1\baselineskip}
\\
ok.





\item
ok.
\vspace*{1\baselineskip}
\\
ok.



\end{enumerate}



\end{enumerate}
\end{document}
